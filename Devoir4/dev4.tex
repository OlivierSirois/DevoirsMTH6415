\documentclass[oneside]{book}
\usepackage[utf8]{inputenc}
\usepackage{float}
\usepackage{graphicx}
\usepackage{amsmath}
\usepackage{color}
\usepackage{multicol}
\usepackage{ragged2e}
\usepackage{listings}
\usepackage{pdfpages}
\title{Devoir 4 \\ MTH 6415}
\date{2018-03-28}
\author{Olivier Sirois - 1626107 \\ Corey Ducharme - 1626614}
\setlength\parindent{0pt}
\makeindex
\pagenumbering{arabic}
\begin{document}
\setcounter{page}{1}
\maketitle
\section*{4.19 (3.19 ed. 4)}
\subsection*{a)}
Dans ce problème, nous avons à chaque étape k deux décision possible se stationner avec une probabilité p ou réessayer plus tard à l'étape k-1. 
Comme montrer dans l'énoncé de b). Une décision optimale est prise si on trouve un stationnement lorsqu'il nous reste seulement $k*$ essaies restants. Nous povons modéliser ce problème sous la forme des équations de Bellmann. Soit $J_k$ le coût à l'étape k :\\


\begin{math}
 J_k(x_k) = 
\begin{cases}
p*k + (1-p)*J_{k-1}(x_{k-1}), & \text{si on essaie de se stationner} \\
J_{k-1}(x_{k-1}), & \text{sinon}
\end{cases}
\end{math}\\

%\begin{math}
%min E \{ J_k(x_k) \} = 
%\begin{cases}
%p*k + (1-p)*E\{J_{k-1}(x_{k-1})\}, & \text{si } E > k*\\
%E\{J_{k-1}(x_{k-1})\}, & \text{si } E < k*
%\end{cases}
%\end{math}\\

Qui est équivalent à :\\

\begin{math}
J_k(x_k) = 
\begin{cases}
pk + (1-p)J_{k-1}(x_{k-1}), & \text{si on essaie de se stationner} \\
pJ_{k-1}(x_{k-1}) + (1-p)J_{k-1}(x_{k-1}), & \text{sinon }
\end{cases}
\end{math}\\

puisque $p + (1-p) = 1$.
\\\\
Posons maintenant $F_k = E\{J_k(x_k)\}$ qui est le coût minimal espéré si on est à k place de stationnement de notre destination. 

\begin{math}
\min E \{ J_k(x_k) \} = 
\min\begin{cases}
pk + (1-p)E\{J_{k-1}(x_{k-1})\}, \\
pE\{J_{k-1}(x_{k-1})\} + (1-p)E\{J_{k-1}(x_{k-1})\},
\end{cases}
\end{math}
\\

En mettant les termes en commun et en remplaçant avec $F_k$, on trouve:

\begin{align*}
\min E \{ J_k(x_k) \} &= p\min(k,E \{J_{k-1}(x_{k-1}\}) + (1-p)*E \{J_{k-1}(x_{k-1})\} \\
F_k &= p\min(k, F_{k-1}) + (1-p)*F_{k-1}
\end{align*}

En définissant $q = (1-p)$, on retrouve l'équation demandée.

\begin{align*}
F_k = p\min(k, F_{k-1}) + qF_{k-1}
\end{align*}



\subsection*{b)}
Pour cette question, on nous demande de confirmer l'optimalité de la politique expliqué dans l'énoncé. Comme nous l'avons vu dans la question a), la fonction $F_k$ utilise le minimum entre k et $F_{k-1}$. Si on réussi à prouver qu'il y a seulement un point d'intersection entre k et $F_{k-1}$, on est capable d'un déduire que la politique aura la forme décrite dans l'énoncé. \\

En utilisant les définitions de la question précédent et en commençant avec $F_0 = C$, on peut dire:\\

\begin{align*}
F_1(1) &= p\min(1, C) + qC\\
F_2(2) &= p\min(2, C) + qF_1(1) = p\min(2,C) + qp\min(1,C) + q^2C\\
F_3(3) &= p\min(3, C) + qF_2(2) = p\min(3,C) + qp\min(2,C) + q^2p\min(1,C) + q^3C\\
\end{align*}
de forme plus générale :\\


\begin{math}
F_k(k) =  
\begin{cases}
q^kC + \sum\limits_{i=1}^{k}iq^{k-i}p, & \text{si } k < C\\
q^kC + \sum\limits_{i=1}^{C}iq^{k-i}p + \sum\limits_{i=C}^{k}Cq^{k-i}p, & \text{si }k > C\\
\end{cases}
\end{math}\\

on peut alors traduire cette forme en différence:

\begin{align*}
F_{k} - F_{k-1} = q^kC - q^{k-1}C + kp 
\end{align*}

Alors,
\begin{align*}
F_0 &= C\\
F_1 &= F_0 + q^1C - q^0C + p\\
F_2 &= F_1 + q^2C - q^1C + 2p\\
...
\end{align*}
Avec un peu de manipulation, on peut voir que pour tout i satisfaisant à l'équation de l'énoncé, $F_i - F_{i-1} < 1$. cela nous assure donc que nous aura seulement une intersection avec $k$ qui correspondera au seuil de la politique. Parallèlement, la valeur d'i ou l'équation dans l'énoncé ne tiendra plus ce qui vérifie l'énoncé.

\section*{4.16 (3.16 de l'ed. 4)}
Pour ce problème, on peut suivre la démarche que nous avons faites dans le cours sur le problème de la vente d'un actif. Évidemment, il y a certaines différences. On peut commencer par définir les décisions possible à chaque étape. $u_k = 1$ si on vend et $\mu_k = 0$ si on ne vend pas. \\\\
On définie maintenant la variable d'état $y_k$ de notre problème qui est la meilleur offre reçue. Étant donné que nous conservons toutes les offres, On peut prend le maximum entre l'offre et $y_{k-1}$ qui est le maximum de toutes les autres offres reçues.

\begin{gather*}
y_k = 
\begin{cases}
\max(y_{k-1}, \omega_k) & P = p_k\\
y_{k-1} & P = (1-p_k)
\end{cases}
\end{gather*}

On peut voir que nous avons deux variables aléatoire dans notre problème, pour ne pas tous les écrire à chaque fois, nous allons définir la variable $\Omega_k(\omega_k, p_k)$ qui est la vecteur aléatoire de notre problème.

On défini maintenant la variable d'état $x_k$ de notre système à la période k+1.

\begin{gather*}
x_k+1 = 
\begin{cases}
y_k &\text{si  } \mu_0 = ... = \mu_k = 1 \\
\Delta &\text{sinon}
\end{cases}
\end{gather*}

Les décisions admissible à une étape k quelconque sont 

\begin{gather*}
\begin{cases}
\mu_k = 0 &\text{si  } x_k = \Delta \\
\mu_k = \{0, 1\} &\text{sinon}
\end{cases}
\end{gather*}

À chaque étape $k > 0$, on perd de l'argent sur la maintenance. Le profit à l'étape k est donc 

\begin{gather*}
g(x_k, \mu_k, \Omega_k) =
\begin{cases}
x_k - kc &\text{si  } \mu_k = 1 \\
0 &\text{sinon}
\end{cases}
\end{gather*}

Soit $J_k(x_k)$, le profit espéré optimal de l'étape k jusqu'à N. Son expression est 

\begin{gather*}
J_k(x_k) = 
\begin{cases}
0 &\text{si  } x_k = \Delta \\
x_N &\text{si  } k = N \text{ et } x_N \neq \Delta \\
\max\left(x_k - kc, E_{\Omega_k}\left[J_{k+1}(y_{k})\right]\right) &\text{sinon}
\end{cases}
\end{gather*}

On peut remplacer la valeur de $x_{k+1}$ par $y_k$ dans la fonction $J_{k+1}$ puisque nous savons que nous n'allons pas vendre. On voie que notre maximum contient à gauche, la décision de vendre et à droite, la décision d'attendre.
\\\\
Ainsi, nous pouvons déterminer que la décision optimale, si on a encore le choix est de vendre si et seulement si 

\begin{gather*}
x_k \geq \alpha_k \equiv E_{\Omega_k}\left[J_{k+1}(y_{k})\right] + kc
\end{gather*}


\section*{1.14 (1.21 de l'ed. 4)}
Pour ce numéro on nous demande de vérifier l'optimalité des politiques décrites en a), b) et c). Nous allons premièrement commencer par généraliser le critère de sélection de $u$ pour chaque étape.\\

étant donné que pour la prise de décision nous avons pas encore l'information pour chacune des valeurs $w$ étant donné sa stochasticité, nous allons généraliser $w_k = \bar{w},$   $ \forall k$

\begin{align*}
J_N(x_N) = x_N \rightarrow \text{aucune décision}\\
J_{N-1}(x_{N-1}) = (1-u_{N-1})*x_{N-1} + J_N(x_{N-1} + x_{N-1}*\bar{w}*u_{N-1}) =\\
J_{N-1}(x_{N-1}) = x_{N-1}(2 + u_{N-1}(\bar{w} - 1))\\
J_{N-2}(x_{N-2}) = (1-u_{N-2})*x_{N-2} + (x_{N-2} + x_{N-2}*u_{N-2}*\bar{w})(2 + u_{N-1}(\bar{w}-1))... 
\end{align*} 
Comme vous pouvez le voir, la fonction J commence déjà à etre très grande. Par contre, nous pouvons analyser la prise de chaque décision à partir de $J_N$. Évidemment, aucune décision ne peut être prise à $J_N$ étant donné que c'est l'état final. Par contre, lorsqu'on remonte à $J_{N-1}$, on peut voir que sa dérivé donne tout simplement $(\bar{w} - 1)$.\\

\subsection*{a)}
Sachant que $\bar{w} > 1$, nous savons que la première décision est $u_{N-1} = 1$ vu que $(\bar{w} - 1)$ est strictement positif. Pour maximiser le terme nous allons donc utiliser $u_{N-1}$ jusqu'à sa limite positive qui est de 1.\\

Évidemment, on ne peut pas confirmer que toute les étapes vont se comporter de cette manière. Cependant, nous pouvons fixer la valeur de $u_{N-1}$ à 1 et le transmettre ensuite dans le terme $J_{N-2}$. Cela nous donnera donc:\\

\begin{align*}
J_{N-1} = x_{N-1}(2 + (1)*(\bar{w} - 1)) = x_{N-1}(1 + \bar{w})\\
J_{N-2} = (1 - u_{N-2})*x_{N-2} + (x_{N-2} + x_{N-2}*u_{N-2}*\bar{w})(1 + \bar{w}) = ...\\
J_{N-2} = x_{N-2}(2 + \bar{w} + u_{N-2}(\bar{w} + \bar{w}^2 - 1))
\end{align*}
Encore une fois, le terme $(\bar{w} + \bar{w}^2 - 1)$ est strictement positif étant donné que $\bar{w} > 1$. En continuant jusqu'à k, on peut ainsi voir que le terme à l'intérieur du $u$ donne:\\

\begin{align*}
J_{N-k} = x_{N-k}(.... + \underline{u_{N-k}(\bar{w}*(\bar{w} + 1)^{k-1} -1)})
\end{align*}
qui est strictement croissant dans le cas de $\bar{w} > 1$, ce qui confirme donc que la décision sera $u_k = 1 ,\forall k$

\subsection*{b)}
Dans le cas de $0 < \bar{w} < \frac{1}{N}$, on peut voir dès le départ que la décision pour $J_{N-1}$ sera 0. De la, nous pouvons repartir de la pour trouver $J_{N-2}$. Donc:\\

\begin{align*}
J_{N-1} = x_N(2)\\
J_{N-2} = x_{N-2}(3 + u_{N-2}(2\bar{w} -1))\\
J_{N-3} = x_{N-3}(4 + u_{N-3}(3\bar{w} -1))...\\
J_{N-k} = x_{N-k}(k+1 + u_{N-k}(k\bar{w} - 1))\\
\&\\
J_{N-N} = J_0 = x_0(N+1 + u_0(N\bar{w} - 1))\\
\end{align*}
On peut voir que tant que $(k\bar{w} - 1) < 0$, nous allons prendre comme décision $u_k = 0$ étant donné que le maximum de $J$ se trouve au point ou $u$ est minimale, soit 0 (sa limite inférieur). Évidemment, on peut voir aussi qu'avec $\bar{w} = \frac{1}{N}$, nous avons au point $J_0$: $(\frac{N}{N} - 1) = 0$, ce qui démontre qu'à la limite supérieur de notre plage pour $\bar{w}$ nous avons une décision indécise, tandis qu'avec n'importe quel valeur de $\bar{w} < \frac{1}{N}$, nous aurons $(k\bar{w} - 1) < 0, \forall k$.\\

\subsection*{c)}
Évidemment, pour cette question nous commençons avec une valeur de $\bar{w} \leq 1$, nous pouvons déjà voir que la valeur de $(\bar{w} - 1)$ sera inférieur ou égale à 0, ce qui implique que notre première décision sera presque assurément $u_{N-1} = 0$. Évidemment, on peut réutiliser notre formule développer en b) pour trouver la valeur de $k'$ qui rendra $(k'\bar{w} - 1) \geq 0 \rightarrow \bar{w} \geq \frac{1}{k'}$.\\

Par contre, le système est discrèt. Alors nous ne pouvons pas simplement utiliser cette formule directement. Par contre, en discrétisant la formule, nous allons nécessairement avoir une valeur $\bar{k}$ dans lequel notre seuil se trouvera entre $\bar{k}$ et $\bar{k}+1$, ce qui nous ramène donc à la forme :\\

$\frac{1}{\bar{k} + 1} < \bar{w} \leq \frac{1}{\bar{k}}$\\

alors, pour les valeurs ou $0 \leq k \leq \bar{k}$, nous aurons $u_{N-k} = 0$, et pour les valeurs ou $N \geq k \geq \bar{k}+1$, nous aurons $u_{N-k} = 1$, ce qui correspond à l'énoncé. 
 
\end{document}