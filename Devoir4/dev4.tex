\documentclass[oneside]{book}
\usepackage[utf8]{inputenc}
\usepackage{float}
\usepackage{graphicx}
\usepackage{amsmath}
\usepackage{color}
\usepackage{multicol}
\usepackage{ragged2e}
\usepackage{listings}
\usepackage{pdfpages}
\title{Devoir 4 MTH 6415}
\date{2018-01-01}
\author{Olivier Sirois 1626107, Corey Ducharme}
\setlength\parindent{0pt}
\makeindex
\pagenumbering{arabic}
\begin{document}
\setcounter{page}{1}
\maketitle
\section*{4.19 (3.19 ed. 4)}
\subsection*{a)}
Comme montrer dans l'énoncé de b). Une décision optimale est prise si on trouve un stationnement lorsqu'il nous reste seulement $k*$ essaies restants. Sachant cela, on peut traduire cette information sous forme:\\


\begin{math}
min E \{ J_k(x_k) \} = 
\begin{cases}
p*k + (1-p)*E\{J_{k-1}(x_{k-1})\}, & \text{si } E > k*\\
E\{J_{k-1}(x_{k-1})\}, & \text{si } E < k*
\end{cases}
\end{math}\\


qui peut être représenté comme étant:\\

\begin{math}
min E \{ J_k(x_k) \} = 
\begin{cases}
p*k + (1-p)*E\{J_{k-1}(x_{k-1})\}, & \text{si } E > k*\\
p*E\{J_{k-1}(x_{k-1})\} + (1-p)E\{J_{k-1}(x_{k-1})\}, & \text{si } E < k*
\end{cases}
\end{math}\\

Étant donnée que $p + (1-p) = 1$. En agrégeant ces deux cas, il est correct de dire:\\

$min E \{ J_k(x_k) \} = p*min(k,E\{J_{k-1}(x_{k-1}\}) + (1-p)*E\{J_{k-1}(x_{k-1}\}$\\

en définissant $minE\{J_k(x_k)\} = F_k$,comme mentionné dans l'énoncé, on revient à l'équation originale, soit:\\

$F_k = p*min(k, F_{k-1}) + (1-p)*F_{k-1}, q = (1-p)$.

\subsection*{b)}

\section*{4.16 (3.16 de l'ed. 4)}
Pour ce problème, on peut se fier à l'example 3.4.1 (ed. 4) du livre de Bertsekas pour commencer. Évidemment, il y a certaines différences. On peut commencer par définir l'état de notre problème:\\

$x_{k+1} = max(x_k, w_k)$\\

Étant donné que nous conservons toutes les offres, On peut dire que l'état serait la meilleur offre que nous avons eu jusqu'à présent. ou $x_k$ est l'offre et que $w_k$ est la prochaine offre que nous recevront. Si l'offre recu est meilleur, elle prendra la place de $x_k$, sinon $x_k$ restera.\\

On peut charactérisé les équations de programmation dynamique comme étant:\\

$V_N(x_N) = x_N)$\\

et:\\

$V_k(x_k) = max[x_k ,E[V_{k+1}(max(x_k,E[\omega_k)]] - c*(N-k) ]$\\

Étant donnée qu'on conserve toutes les offres déjà eu auparavant, on peut assumer que l'ensemble $T$ est absorbant. l'offre effective ne peut pas diminuer, alors selon le 'one step lookahed stopping rule', il s'ensuit que la première offre qui est égale ou excède $\bar{\alpha}$ est optimale. Il est ensuite logique d'assumer que:\\

$\alpha_k = E\{max(x_{k+1},\omega_{k+1})\} - c*(N-k)$\\

ou la limite asymptotique de $\alpha_k$ qui peut être défini comme étant:\\

$\bar{\alpha} = P(\bar{\alpha})\bar{\alpha} + \int\limits_{\bar{\alpha}}^{\infty}\omega dP(\omega)$
\end{document}